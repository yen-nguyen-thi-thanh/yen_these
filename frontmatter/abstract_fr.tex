%!TEX root = ../dissertation.tex
% the abstract
\selectlanguage{french}

\noindent{\large{\ROBCB Résumé}}\\

Cette thèse présente les applications de la théorie du Transport Optimal et des statistiques dans deux domaines : la biologie et l'actuariat. Nous apprenons la divergence de Sinkhorn, une classe de divergences entre objets, basée sur la distance OT régularisée pour découvrir les modèles des ensembles de données. La divergence de Sinkhorn est considérée comme la fonction de perte que nous voulons minimiser en utilisant une descente de gradient en mini-batch et l'algorithme de Sinkhorn.\\
Dans la première partie de cette thèse, nous présentons plusieurs algorithmes conçus pour apprendre un motif de correspondance entre deux ensembles de données dans des situations où il est souhaitable de faire correspondre des éléments qui présentent une relation appartenant à un modèle paramétrique connu. Les algorithmes se déroulent en deux étapes. Premièrement, un plan de transport optimal et une transformation affine optimale sont appris. Ensuite, la matrice OT est exploitée pour dériver soit plusieurs co-clusters, soit plusieurs ensembles d'éléments appariés.\\
Dans la deuxième partie, nous développons une nouvelle méthodologie pour anticiper les villes qui demanderont une déclaration de catastrophe naturelle pour un événement de sécheresse, une étape clé du système d'indemnisation national. Nous construisons un algorithme proximal inertiel pour l'optimisation non convexe. Le problème d'optimisation est conçu de manière à produire un vecteur clairsemé de prédictions car on sait que relativement peu de villes feront la demande.


\noindent {\ROBCB Mots-Clefs :} Algorithme de Sinkhorn; contraste de Sinkhorn; co-clustering spectral; g\'enomique; maladie de Huntington; matching; transport optimal.

\selectlanguage{english}


