%!TEX root = ../dissertation.tex
% the abstract
\noindent {\large{\ROBCB Abstract}}\\

Optimal transport theory has found many application in diverse fields in machine learning thank to 
providing a powerful tool for comparing probability distributions, one of a crucial issue
in machine learning.  In this thesis, we leverage the optimal transport theory and statistics to deal with 
the problems in biology and actuary. We develop new methodology based on evaluating OT distance
between empirical distributions attached on real datasets and turn it into a loss function for optimisation
using a mini-batch gradient descent and Sinkhorn algorithm.  


%This thesis presents the applications of the Optimal Transport theory and  
%statistics in two domains: biology and actuary. We learn the Sinkhorn 
%divergence, a classe of discrepancies between objects, based on regularized
%OT distance to discover the patterns of datasets. The Sinkhorn divergence is 
%considered as the loss function which we want to minimize using a mini-batch
%gradient descent and Sinkhorn algorithm.  \\
In the first part of this thesis, we  present  several  algorithms  designed   to
learn  a  pattern  of correspondence between two data sets in  situations where
it is desirable to match elements that  exhibit a relationship belonging to  a known parametric
model.  The algorithms unfold  in two stages.  First, an  optimal transport plan
and an optimal  affine transformation are learned. Second,  the OT matrix is  exploited to
derive either several co-clusters or several sets of matched elements.\\
In the second part,  we develop a new methodology to anticipate which cities will request
a declaration  of natural disaster  for a drought event,  a key step  of the
national compensation  scheme. We build an  inertial proximal algorithm for  nonconvex optimization.  
The optimisation  problem  is  designed  so  as to  yield  a  sparse  vector  of
predictions because  it is known  that relatively  few cities will  make the
request.

  
  \textbf{Keywords.} Optimal transport; Sinkhorn algorithm; Sinkhorn divergence;
  proximal algorithm; matching; Huntington's disease; omics data; natural disasters.
  
       
  