\section{What this thesis is about?}

Optimal transport theory has found many application in diverse fields in machine learning thank to 
providing a powerful tool for comparing probability distributions, one of a crucial issue
in machine learning.  In this thesis, we leverage the optimal transport theory and statistics to deal with 
the problems in biology and actuary. The biological problem is to assess the possible relationships between microRNA and mRNA expression in the striatum of Huntington's disease model mice. The actuariat problem relates to anticipate the declaration of natural disaster for a drought event. 

\paragraph{Micro-RNA regulation in the striatum of Huntington’s disease model mice}
Huntington's disease is an autosomal-dominant, progressive neurodegenerative disorder characterized by involuntary choreatic movements with cognitive and behavioral disturbances. Currently there are no therapies to prevent the onset or slow the progression of HD.  Huntington disease (HD) is  caused by an expansion of  a repeating CAG triplet series in the huntingtin gene. Like several neurodegenerative diseases such as Alzheimer's disease, Parkinson's disease and amyotrophic lateral sclerosis, Huntington's disease relates to gene deregulation which has encouraged large studies to gene regulatory mechanisms at different level. The most important instruments to adjust gene expression at the post-transcriptional level are small non-coding RNAs called miRNAs. \yen{rewrite: How do miRNAs regulate gene expression? In most cases, microRNA controls gene expression mainly by binding with messenger RNA (mRNA) in the cell cytoplasm. Instead of being translated quickly into a protein, the marked mRNA will be either destroyed and its components recycled, or it will be preserved and translated later.} Our ultimate goal is to study the interaction between mRNAs and miRNAs in the Huntington disease. 

\paragraph{Anticipate the declaration of natural disaster for a drought event in France}




\section{Formalisation}

\paragraph{Micro-RNA regulation in the striatum of Huntington’s disease model mice}
To shed light the relationship between mRNAs and miRNAs, we analyse the miRNA and mRNA data collected at three different ages in striatum (a brain region) from an allelic series of HD model knock-in mice with increasing CAG length in the endogenous Huntingtin gene. For each combination of poly Q length and age, we have quantified miRNA and mRNA expression of 8 mice including 4 females and 4 males. After preprocessing ...we obtained the final dataset consisting of $M= 13 616$ mRNA profiles and $N=1143$ miRNA profile. The fact that miRNA and target mRNAs are linked by a "many-to-many" mirroring relationship because a miRNA induces the degradation of a target mRNA or blocks its translation into proteins, or both and a miRNA can regulate several mRNAs. 

\section{State of the art}

\paragraph{Micro-RNA regulation in the striatum of Huntington’s disease model mice}

\paragraph{Anticipate declaration of natural disaster for a drought event}


\section{Design, programming and implementation of algorithms}
\section{Results}
