\section{What this thesis is about?}

Optimal transport theory has found many application in diverse fields in machine learning thank to 
providing a powerful tool for comparing probability distributions, one of a crucial issue
in machine learning.  In this thesis, we leverage the optimal transport theory and statistics to deal with 
the problems in biology and actuary. The biological problem is to assess the possible relationships between microRNA and mRNA expression in the striatum of Huntington's disease model mice. The actuariat problem relates to anticipate the declaration of natural disaster for a drought event. 

\paragraph{Micro-RNA regulation in the striatum of Huntington’s disease model mice}
Huntington's disease is an autosomal-dominant, progressive neurodegenerative disorder caused by an expansion of  a repeating CAG triplet series in the huntingtin gene. Huntington disease (HD) is characterized by involuntary choreatic movements with cognitive and behavioral disturbances. Currently there are no therapies to prevent the onset or slow the progression of HD. 
 Like several neurodegenerative diseases such as Alzheimer's disease, Parkinson's disease and amyotrophic lateral sclerosis, Huntington's disease relates to gene deregulation 

\paragraph{Anticipate declaration of natural disaster for a drought event}

\section{Formalisation}
\section{Design, programming and implementation of algorithms}
\section{Results}
